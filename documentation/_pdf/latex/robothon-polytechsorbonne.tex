%% Generated by Sphinx.
\def\sphinxdocclass{report}
\documentclass[letterpaper,10pt,english]{sphinxmanual}
\ifdefined\pdfpxdimen
   \let\sphinxpxdimen\pdfpxdimen\else\newdimen\sphinxpxdimen
\fi \sphinxpxdimen=.75bp\relax
\ifdefined\pdfimageresolution
    \pdfimageresolution= \numexpr \dimexpr1in\relax/\sphinxpxdimen\relax
\fi
%% let collapsible pdf bookmarks panel have high depth per default
\PassOptionsToPackage{bookmarksdepth=5}{hyperref}

\PassOptionsToPackage{booktabs}{sphinx}
\PassOptionsToPackage{colorrows}{sphinx}

\PassOptionsToPackage{warn}{textcomp}
\usepackage[utf8]{inputenc}
\ifdefined\DeclareUnicodeCharacter
% support both utf8 and utf8x syntaxes
  \ifdefined\DeclareUnicodeCharacterAsOptional
    \def\sphinxDUC#1{\DeclareUnicodeCharacter{"#1}}
  \else
    \let\sphinxDUC\DeclareUnicodeCharacter
  \fi
  \sphinxDUC{00A0}{\nobreakspace}
  \sphinxDUC{2500}{\sphinxunichar{2500}}
  \sphinxDUC{2502}{\sphinxunichar{2502}}
  \sphinxDUC{2514}{\sphinxunichar{2514}}
  \sphinxDUC{251C}{\sphinxunichar{251C}}
  \sphinxDUC{2572}{\textbackslash}
\fi
\usepackage{cmap}
\usepackage[T1]{fontenc}
\usepackage{amsmath,amssymb,amstext}
\usepackage[francais]{babel}



\usepackage[T1]{fontenc}
\usepackage{lmodern}



\usepackage[Bjornstrup]{fncychap}
\usepackage[,numfigreset=1,mathnumfig]{sphinx}

\fvset{fontsize=auto}
\usepackage{geometry}


% Include hyperref last.
\usepackage{hyperref}
% Fix anchor placement for figures with captions.
\usepackage{hypcap}% it must be loaded after hyperref.
% Set up styles of URL: it should be placed after hyperref.
\urlstyle{same}

\addto\captionsenglish{\renewcommand{\contentsname}{Projet}}

\usepackage{sphinxmessages}
\setcounter{tocdepth}{1}

\usepackage{graphicx}

\title{Robothon \sphinxhyphen{} Polytech Sorbonne}
\date{Jun 06, 2024}
\release{0.1}
\author{Author}
\newcommand{\sphinxlogo}{\vbox{}}
\renewcommand{\releasename}{Release}
\makeindex
\begin{document}

\ifdefined\shorthandoff
  \ifnum\catcode`\=\string=\active\shorthandoff{=}\fi
  \ifnum\catcode`\"=\active\shorthandoff{"}\fi
\fi

\pagestyle{empty}
\sphinxmaketitle
\pagestyle{plain}
\sphinxtableofcontents
\pagestyle{normal}
\phantomsection\label{\detokenize{index::doc}}


\sphinxAtStartPar
Bienvenue sur la documentation du projet industriel “robothon”.

\sphinxAtStartPar
Projet dans le cadre du module projet industriel de la spécialité robotique.

\sphinxAtStartPar
\sphinxstyleemphasis{Etudiants} : DAI Yannick, DAT Mathieu, RAZANAJATOVO Faliana, TEIXEIRA Pierre

\sphinxAtStartPar
\sphinxstyleemphasis{Clients} : KHORAMSHAHI Mahdi, DONCIEUX Stéphane

\sphinxAtStartPar
\sphinxstyleemphasis{Tuteurs académiques} : BAUDRY Aline, PLUMET Frédéric

\sphinxstepscope


\chapter{Vue d’ensemble du projet}
\label{\detokenize{vue-ensemble:vue-d-ensemble-du-projet}}\label{\detokenize{vue-ensemble::doc}}


\sphinxstepscope


\chapter{Contexte du projet}
\label{\detokenize{contexte:contexte-du-projet}}\label{\detokenize{contexte::doc}}














\sphinxstepscope


\chapter{Approche pour mener à bien le projet}
\label{\detokenize{approche:approche-pour-mener-a-bien-le-projet}}\label{\detokenize{approche::doc}}
\sphinxAtStartPar
\sphinxstylestrong{Ressources}
\begin{itemize}
\item {} 
\sphinxAtStartPar
\sphinxurl{https://github.com/peterso/robothon-grand-challenge?tab=readme-ov-file}

\item {} 
\sphinxAtStartPar
\sphinxurl{https://wiki.ros.org/}

\end{itemize}

\sphinxAtStartPar
ce projet industriel a été pour nous une opportunité d’aprendre à utiliser une plateforme telle que Docker pour faciliter l’organisation du travail dans une équipe, mais aussi pour nous familiariser avec l’OS Linux, et être habitué à l’utilisation du terminale pour l’exécution et la compilation de programmes, mais aussi la découverte d’un nouvel outil, très intéressant et assez important dans la robotique car la majorité des systèmes robotisés l’utilise (on parle ici de ROS).

\sphinxAtStartPar
Comme il s’agit d’une découverte, on a passé une bonne partie de notre temps à se documenter et à apprendre. Pour mener à bien le projet, on s’est basé sur les différentes solutions développées par les équipes des éditions précédentes de la compétition Robothon, pour s’y inspirer, en prenant les bonnes solutions, en apprenant de leurs erreurs et en optimisant les raisonnements.

\sphinxAtStartPar
A travers 3 éditions, on a pu remarquer que l’équipe de “\sphinxstyleemphasis{Platonics}”, dont les codes et modules étaient les mieux documentés et rendus open\sphinxhyphen{}source à 100\%, avait développé des solution très intéressantes, qu’on a pas mal utilisé comme base pour les différentes tâches à accomplir comme la localisation de la task\sphinxhyphen{}board avec SIFT.

\sphinxAtStartPar
Dans la dynamique du travail collaboratif, on a créé un repository sur gitlabsu (\sphinxurl{https://gitlabsu.sorbonne-universite.fr/robothon-sorbonne}) avec différents packages en fonction des tâches et des différents installations qu’il fallait qu’on fasse pour s’harmoniser au niveau du groupe.

\sphinxAtStartPar
\sphinxstylestrong{Installation de l’environnement Docker avec ROS et tous les packages nécessaires}

\sphinxAtStartPar
Logiciels installés (incluant environnement, et OS à avoir) :
\begin{itemize}
\item {} 
\sphinxAtStartPar
Linux (Ubuntu, peu importe la version car on va utiliser Docker),

\item {} 
\sphinxAtStartPar
Docker (pour pouvoir contenir ROS Noetic qui n’est utilisable que sous Ubuntu version 20.04),

\item {} 
\sphinxAtStartPar
ROS1 Noetic (”\sphinxstyleemphasis{Robotic Operating System}” installé dans Docker),

\item {} 
\sphinxAtStartPar
Terminator (gestionnaire de terminal pour faciliter la compilation de plusieurs programmes en ayant un oeil sur les autres en train de tourner)

\end{itemize}

\sphinxAtStartPar
\sphinxstylestrong{Qu’est\sphinxhyphen{}ce que “Docker” ?}

\sphinxAtStartPar
Docker est une plateforme open source qui automatise le déploiement d’applications dans des conteneurs légers. Un conteneur regroupe l’application et toutes ses dépendances dans un seul paquet, garantissant que l’application fonctionne de manière cohérente, quel que soit l’environnement, donc indépendante à l’OS dans lequel il est. Docker permet de créer, déployer et gérer ces conteneurs de manière efficace, offrant ainsi une solution portable et flexible pour le développement et le déploiement d’applications. Il simplifie également la gestion des versions et des configurations, facilitant la collaboration entre les équipes de développement et d’exploitation.

\sphinxAtStartPar
Dans notre cas, on l’utilise pour la gestion de version, donc pouvoir utiliser ROS Noetic même avec un OS Ubuntu version 22.04, qui de base n’est pas compatible avec cette version de ROS, mais on utilise aussi Docker pour harmoniser les environnements sur chacun de nos appareils (les ordinateurs auront donc la même configuration et les mêmes pachages une fois que l’image Docker est compilée).

\sphinxAtStartPar
\sphinxstylestrong{Commandes pour accéder aux différents packages qui ont été développés :}

\sphinxAtStartPar
Installation de Docker incluant déjà tous les packages liés au projet :

\begin{sphinxVerbatim}[commandchars=\\\{\}]
\PYGZdl{}\PYG{+w}{ }git\PYG{+w}{ }clone\PYG{+w}{ }https://gitlabsu.sorbonne\PYGZhy{}universite.fr/robothon\PYGZhy{}sorbonne/docker.git\PYG{+w}{ }\PYGZhy{}\PYGZhy{}recursive
\end{sphinxVerbatim}

\sphinxAtStartPar
Se placer dans le répertoire docker :

\begin{sphinxVerbatim}[commandchars=\\\{\}]
\PYGZdl{}\PYG{+w}{ }\PYG{n+nb}{cd}\PYG{+w}{ }docker
\end{sphinxVerbatim}

\sphinxAtStartPar
Installation de l’image Docker :

\begin{sphinxVerbatim}[commandchars=\\\{\}]
\PYGZdl{}\PYG{+w}{ }./install\PYGZus{}docker.bash
\end{sphinxVerbatim}

\sphinxAtStartPar
Une fois l’installation de l’image, il suffit de lancer Docker :

\begin{sphinxVerbatim}[commandchars=\\\{\}]
\PYGZdl{}\PYG{+w}{ }./start\PYGZus{}docker.sh
\end{sphinxVerbatim}

\sphinxAtStartPar
Pour avoir accès à terminator, il suffit de saisir :

\begin{sphinxVerbatim}[commandchars=\\\{\}]
\PYGZdl{}\PYG{+w}{ }terminator
\end{sphinxVerbatim}

\sphinxstepscope


\chapter{Environnement logiciel}
\label{\detokenize{environnement:environnement-logiciel}}\label{\detokenize{environnement::doc}}
\sphinxAtStartPar
Ceci présente l’environnement logiciel : ROS, Docker, git, gitlab, etc.

\sphinxAtStartPar
Est\sphinxhyphen{}ce que la compilation de schéma avec \sphinxstyleemphasis{Mermaid} fonctionne en \sphinxstyleemphasis{LaTeX} ?

\begin{figure}[htbp]
\centering
\capstart
\caption{Ceci est une figure centrée avec une légende}\label{\detokenize{environnement:id1}}\end{figure}

\begin{figure}[htbp]
\centering
\capstart
\caption{Graphe pour l’exécution d’une tâche}\label{\detokenize{environnement:id2}}\end{figure}

\begin{figure}[htbp]
\centering
\capstart
\caption{Superviseur}\label{\detokenize{environnement:id3}}\end{figure}

\sphinxAtStartPar
!(./img/new/Graphe.png)







\sphinxstepscope


\chapter{Localisation de la task\sphinxhyphen{}board à l’aide de SIFT}
\label{\detokenize{SIFT:localisation-de-la-task-board-a-l-aide-de-sift}}\label{\detokenize{SIFT::doc}}


























\begin{sphinxVerbatim}[commandchars=\\\{\}]
class\PYG{+w}{ }LocalizationService\PYG{o}{(}\PYG{o}{)}\PYG{o}{\PYGZob{}}
\PYG{+w}{    }func\PYG{+w}{ }\PYGZus{}\PYGZus{}init\PYGZus{}\PYGZus{}\PYG{o}{(}self\PYG{o}{)}
\PYG{+w}{    }func\PYG{+w}{ }compute\PYGZus{}localization\PYGZus{}in\PYGZus{}pixels\PYG{o}{(}self,\PYG{+w}{ }img:\PYG{+w}{ }Image\PYG{o}{)}
\PYG{+w}{    }func\PYG{+w}{ }publish\PYGZus{}annoted\PYGZus{}image\PYG{o}{(}self\PYG{o}{)}
\PYG{+w}{    }func\PYG{+w}{ }handle\PYGZus{}request\PYG{o}{(}self,\PYG{+w}{ }req\PYG{o}{)}
\PYG{+w}{    }func\PYG{+w}{ }run\PYG{o}{(}self\PYG{o}{)}
\PYG{o}{\PYGZcb{}}
\end{sphinxVerbatim}





\sphinxstepscope


\chapter{Tâche pour appuyer sur le bouton bleu}
\label{\detokenize{bouton-bleu:tache-pour-appuyer-sur-le-bouton-bleu}}\label{\detokenize{bouton-bleu::doc}}


\begin{sphinxVerbatim}[commandchars=\\\{\}]
class\PYG{+w}{ }Panda\PYGZus{}R\PYG{o}{(}Panda\PYG{o}{)}:
\PYG{+w}{    }def\PYG{+w}{ }\PYGZus{}\PYGZus{}init\PYGZus{}\PYGZus{}\PYG{o}{(}self\PYG{o}{)}:
\PYG{+w}{        }super\PYG{o}{(}\PYG{o}{)}.\PYGZus{}\PYGZus{}init\PYGZus{}\PYGZus{}\PYG{o}{(}\PYG{o}{)}
\PYG{+w}{        }self.listener\PYG{+w}{ }\PYG{o}{=}\PYG{+w}{ }tf.TransformListener\PYG{o}{(}\PYG{o}{)}
\PYG{+w}{    }
\PYG{+w}{    }\PYG{c+c1}{\PYGZsh{} Méthode permettant de placer l\PYGZsq{}effecteur final à une hauteur de 0.25m }
\PYG{+w}{    }\PYG{c+c1}{\PYGZsh{} par défaut (mais ci\PYGZhy{}dessous, saisir dans le terminal \PYGZdq{}home\PYGZdq{} le place à 0.5m)}

\PYG{+w}{    }def\PYG{+w}{ }home\PYG{o}{(}self,\PYG{+w}{ }\PYG{n+nv}{z}\PYG{o}{=}\PYG{l+m}{0}.25\PYG{o}{)}:
\PYG{+w}{        }\PYG{n+nv}{pos\PYGZus{}array}\PYG{+w}{ }\PYG{o}{=}\PYG{+w}{ }np.array\PYG{o}{(}\PYG{o}{[}\PYG{l+m}{0}.4,\PYG{+w}{ }\PYG{l+m}{0},\PYG{+w}{ }z\PYG{o}{]}\PYG{o}{)}
\PYG{+w}{        }\PYG{n+nv}{quat}\PYG{+w}{ }\PYG{o}{=}\PYG{+w}{ }np.quaternion\PYG{o}{(}\PYG{l+m}{0},\PYG{+w}{ }\PYG{l+m}{1},\PYG{+w}{ }\PYG{l+m}{0},\PYG{+w}{ }\PYG{l+m}{0}\PYG{o}{)}
\PYG{+w}{        }\PYG{n+nv}{goal}\PYG{+w}{ }\PYG{o}{=}\PYG{+w}{ }array\PYGZus{}quat\PYGZus{}2\PYGZus{}pose\PYG{o}{(}pos\PYGZus{}array,\PYG{+w}{ }quat\PYG{o}{)}
\PYG{+w}{        }goal.header.seq\PYG{+w}{ }\PYG{o}{=}\PYG{+w}{ }\PYG{l+m}{1}
\PYG{+w}{        }goal.header.stamp\PYG{+w}{ }\PYG{o}{=}\PYG{+w}{ }rospy.Time.now\PYG{o}{(}\PYG{o}{)}

\PYG{+w}{        }\PYG{n+nv}{ns\PYGZus{}msg}\PYG{+w}{ }\PYG{o}{=}\PYG{+w}{ }\PYG{o}{[}\PYG{l+m}{0},\PYG{+w}{ }\PYG{l+m}{0},\PYG{+w}{ }\PYG{l+m}{0},\PYG{+w}{ }\PYGZhy{}2.4,\PYG{+w}{ }\PYG{l+m}{0},\PYG{+w}{ }\PYG{l+m}{2}.4,\PYG{+w}{ }\PYG{l+m}{0}\PYG{o}{]}
\PYG{+w}{        }self.go\PYGZus{}to\PYGZus{}pose\PYG{o}{(}goal\PYG{o}{)}
\PYG{+w}{        }self.set\PYGZus{}configuration\PYG{o}{(}ns\PYGZus{}msg\PYG{o}{)}
\PYG{+w}{        }self.set\PYGZus{}K.update\PYGZus{}configuration\PYG{o}{(}\PYG{o}{\PYGZob{}}\PYG{l+s+s2}{\PYGZdq{}nullspace\PYGZus{}stiffness\PYGZdq{}}:10\PYG{o}{\PYGZcb{}}\PYG{o}{)}

\PYG{+w}{        }rospy.sleep\PYG{o}{(}rospy.Duration\PYG{o}{(}\PYG{n+nv}{secs}\PYG{o}{=}\PYG{l+m}{1}\PYG{o}{)}\PYG{o}{)}

\PYG{+w}{        }self.set\PYGZus{}K.update\PYGZus{}configuration\PYG{o}{(}\PYG{o}{\PYGZob{}}\PYG{l+s+s2}{\PYGZdq{}nullspace\PYGZus{}stiffness\PYGZdq{}}:0\PYG{o}{\PYGZcb{}}\PYG{o}{)}
\end{sphinxVerbatim}



\begin{sphinxVerbatim}[commandchars=\\\{\}]
\PYGZdl{}\PYG{+w}{ }Position\PYG{+w}{ }:\PYG{+w}{ }home
\end{sphinxVerbatim}





\begin{sphinxVerbatim}[commandchars=\\\{\}]
\PYGZdl{}\PYG{+w}{ }Position\PYG{+w}{ }:\PYG{+w}{ }home2
\end{sphinxVerbatim}









\begin{sphinxVerbatim}[commandchars=\\\{\}]
def\PYG{+w}{ }run\PYG{o}{(}self\PYG{o}{)}:
\PYG{+w}{        }\PYG{k}{while}\PYG{+w}{ }not\PYG{+w}{ }rospy.is\PYGZus{}shutdown\PYG{o}{(}\PYG{o}{)}:
\PYG{+w}{            }\PYG{n+nv}{position}\PYG{+w}{ }\PYG{o}{=}\PYG{+w}{ }input\PYG{o}{(}\PYG{l+s+s2}{\PYGZdq{}Position : \PYGZdq{}}\PYG{o}{)}
\PYG{+w}{            }\PYG{k}{if}\PYG{+w}{ }position\PYG{+w}{ }\PYG{k}{in}\PYG{+w}{ }self.keys.keys\PYG{o}{(}\PYG{o}{)}:
\PYG{+w}{                }\PYG{n+nv}{char}\PYG{o}{=}self.keys\PYG{o}{[}position\PYG{o}{]}
\PYG{+w}{                }\PYG{k}{if}\PYG{+w}{ }\PYG{n+nv}{char}\PYG{+w}{ }\PYG{o}{=}\PYG{o}{=}\PYG{+w}{ }\PYG{l+s+s1}{\PYGZsq{}home2\PYGZsq{}}:
\PYG{+w}{                    }self.panda.home\PYG{o}{(}\PYG{l+m}{0}.25\PYG{o}{)}
\PYG{+w}{                }\PYG{k}{elif}\PYG{+w}{ }\PYG{n+nv}{char}\PYG{+w}{ }\PYG{o}{=}\PYG{o}{=}\PYG{+w}{ }\PYG{l+s+s1}{\PYGZsq{}home\PYGZsq{}}:
\PYG{+w}{                    }self.panda.home\PYG{o}{(}\PYG{l+m}{0}.5\PYG{o}{)}
\PYG{+w}{                    }self.localization\PYG{o}{(}\PYG{o}{)}
\PYG{+w}{                }\PYG{k}{elif}\PYG{+w}{ }\PYG{n+nv}{char}\PYG{+w}{ }\PYG{o}{=}\PYG{o}{=}\PYG{+w}{ }\PYG{l+s+s2}{\PYGZdq{}quit\PYGZdq{}}:\PYG{+w}{  }\PYG{c+c1}{\PYGZsh{} STOP}
\PYG{+w}{                    }rospy.signal\PYGZus{}shutdown\PYG{o}{(}\PYG{l+s+s2}{\PYGZdq{}User initiated shutdown\PYGZdq{}}\PYG{o}{)}
\PYG{+w}{                }\PYG{k}{elif}\PYG{+w}{ }\PYG{n+nv}{char}\PYG{o}{=}\PYG{o}{=}\PYG{l+s+s1}{\PYGZsq{}test\PYGZsq{}}:
\PYG{+w}{                }
\PYG{+w}{                    }\PYG{n+nv}{pose}\PYG{+w}{ }\PYG{o}{=}\PYG{+w}{ }PoseStamped\PYG{+w}{ }\PYG{o}{(}\PYG{o}{)}
\PYG{+w}{                    }pose.pose.position.x\PYG{+w}{ }\PYG{o}{=}\PYG{+w}{ }\PYG{l+m}{0}
\PYG{+w}{                    }pose.pose.position.y\PYG{+w}{ }\PYG{o}{=}\PYG{+w}{ }\PYG{l+m}{0}.2
\PYG{+w}{                    }pose.pose.position.x\PYG{+w}{ }\PYG{o}{=}\PYG{+w}{ }\PYG{l+m}{0}

\PYG{+w}{                    }rospy.loginfo\PYG{o}{(}pose\PYG{o}{)}
\PYG{+w}{                    }self.panda.go\PYGZus{}to\PYGZus{}pose\PYGZus{}EE\PYG{o}{(}pose\PYG{o}{)}
\PYG{+w}{                }\PYG{k}{elif}\PYG{+w}{ }\PYG{n+nv}{char}\PYG{o}{=}\PYG{o}{=}\PYG{l+s+s1}{\PYGZsq{}circle\PYGZsq{}}:
\PYG{+w}{                    }\PYG{c+c1}{\PYGZsh{} theta goes from 0 to 2pi}
\PYG{+w}{                    }\PYG{n+nv}{theta}\PYG{+w}{ }\PYG{o}{=}\PYG{+w}{ }np.linspace\PYG{o}{(}\PYG{l+m}{0},\PYG{+w}{ }\PYG{l+m}{2}*np.pi,\PYG{+w}{ }\PYG{l+m}{2000}\PYG{o}{)}

\PYG{+w}{                    }\PYG{c+c1}{\PYGZsh{} the radius of the circle}
\PYG{+w}{                    }\PYG{n+nv}{r}\PYG{+w}{ }\PYG{o}{=}\PYG{+w}{ }\PYG{l+m}{0}.1

\PYG{+w}{                    }\PYG{n+nv}{x}\PYG{+w}{ }\PYG{o}{=}\PYG{+w}{ }r*np.cos\PYG{o}{(}theta\PYG{o}{)}\PYG{+w}{ }+\PYG{+w}{ }self.panda.curr\PYGZus{}pos\PYG{o}{[}\PYG{l+m}{0}\PYG{o}{]}
\PYG{+w}{                    }\PYG{n+nv}{y}\PYG{+w}{ }\PYG{o}{=}\PYG{+w}{ }r*np.sin\PYG{o}{(}theta\PYG{o}{)}\PYG{+w}{ }+\PYG{+w}{ }self.panda.curr\PYGZus{}pos\PYG{o}{[}\PYG{l+m}{1}\PYG{o}{]}
\PYG{+w}{                    }
\PYG{+w}{                    }\PYG{n+nv}{z}\PYG{+w}{ }\PYG{o}{=}\PYG{+w}{ }np.full\PYG{o}{(}\PYG{o}{(}\PYG{l+m}{2000}\PYG{o}{)},\PYG{+w}{ }self.panda.curr\PYGZus{}pos\PYG{o}{[}\PYG{l+m}{2}\PYG{o}{]}\PYG{o}{)}
\PYG{+w}{                    }self.panda.set\PYGZus{}stiffness\PYG{o}{(}\PYG{l+m}{4000},\PYG{+w}{ }\PYG{l+m}{4000},\PYG{+w}{ }\PYG{l+m}{4000},\PYG{+w}{ }\PYG{l+m}{50},\PYG{+w}{ }\PYG{l+m}{50},\PYG{+w}{ }\PYG{l+m}{50},\PYG{+w}{ }\PYG{l+m}{10}\PYG{o}{)}

\PYG{+w}{                    }self.panda.play\PYGZus{}trajectory\PYG{o}{(}x,\PYG{+w}{ }y,\PYG{+w}{ }z\PYG{o}{)}
\PYG{+w}{                }\PYG{k}{elif}\PYG{+w}{ }\PYG{n+nv}{char}\PYG{o}{=}\PYG{o}{=}\PYG{l+s+s1}{\PYGZsq{}door\PYGZsq{}}:
\PYG{+w}{                    }self.door\PYG{o}{(}\PYG{o}{)}
\PYG{+w}{                }\PYG{k}{else}:
\PYG{+w}{                    }self.panda.go\PYGZus{}to\PYGZus{}frame\PYG{o}{(}char\PYG{o}{[}\PYG{l+m}{0}\PYG{o}{]},\PYG{+w}{ }\PYG{l+s+s1}{\PYGZsq{}tool\PYGZsq{}},\PYG{+w}{ }char\PYG{o}{[}\PYG{l+m}{1}\PYG{o}{]}\PYG{o}{)}
\PYG{+w}{                    }
\PYG{+w}{            }self.rate.sleep\PYG{o}{(}\PYG{o}{)}
\end{sphinxVerbatim}

















\sphinxstepscope


\chapter{Tâche pour le déplacement du slider en fonction d’un retour visuel}
\label{\detokenize{slider:tache-pour-le-deplacement-du-slider-en-fonction-d-un-retour-visuel}}\label{\detokenize{slider::doc}}


































\sphinxstepscope


\chapter{Ouverture de la trappe avant le prélèvement de la sonde}
\label{\detokenize{trappe:ouverture-de-la-trappe-avant-le-prelevement-de-la-sonde}}\label{\detokenize{trappe::doc}}
\sphinxstepscope


\chapter{Tâche de la sonde pour un prélèvement au niveau d’un capteur}
\label{\detokenize{sonde:tache-de-la-sonde-pour-un-prelevement-au-niveau-d-un-capteur}}\label{\detokenize{sonde::doc}}


\renewcommand{\indexname}{Index}
\printindex
\end{document}